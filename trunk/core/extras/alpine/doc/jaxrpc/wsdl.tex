\subsection{WSDL: an extra complication}
\subsubsection{Automatically Generating WSDL from Java}
\label{objections:o-x:wsdl-gen}

The role of an Interface Definition Language (IDL) has always been
twofold. The primary output of the process is a definition of \emph{the
interface} of the remote system. This is "interface" in the sense of
\emph{the implementation independent signature of the service}, and does
not imply that the implementation language needs an explicit notion of
interfaces. The interface is inherently implementation independent, and
can be frozen or carefully managed with respects to versioning.

The act of writing the IDL inherently forces the author to define 
portable datatypes and operations in its restricted language of the IDL.
This effectively guarantees portability. The implementation languages
invariably contain constructs which are not portable, but as these are
excluded from the Interface language, a portability issue is the
exception, rather than the rule.

Yet the generally accepted practise for working with JAX-RPC discards
all these notions. Instead of generating implementation classes from the
WSDL, the WSDL is often generated from the implementation classes, using
Java's Reflection API. We shall term this process \emph{WSDL-last development}.

This has the following consequences. 

\begin{itemize}

\item
    There is no way to ensure that the published interface of a service
    remains constant over time. Every redeployment of the service, every
    upgrade of the SOAP stack or even the underlying Java runtime may
    change the WSDL.

\item

Some aspects of the service are not extracted from the raw signatures
of the classes and methods. For example, if a method chooses to extract
attachments from the message, that information can be hidden in the contents
of the message, instead of its signature. The generated WSDL will thus
omit any information about the attachment needs of the service.

\item

There is no warning of portability issues before integration time. 
When defining a service using an IDL, the author knows
when there are problems, as the IDL will not compile. Yet with WSDL-last
development, everything seems to work, until the service goes live and
a customer using a different language attempts to import the WSDL and
invoke the service.

\item

Not enough people have been hand-generating WSDL for the WSDL-to-Java
side of the toolchain to be broadly tested as well as it should.

    
\end{itemize}

The alternative to WSDL-last development is clearly, \emph{WSDL-first
development}. Although this is the better approach from the perspective
of portability and interface stability, Web Service developers are not
pushed in this direction.

One of the underlying causes of this has to be the sheer complexity of
XML Schema and WSDL. The XSD type system bears minimal resemblence to
that of current object oriented languages, and WSDL itself is
horrendously over-verbose and under-readable. As evidence of this, 
consider the broad variety of products that aim to make authoring XSD
and WSDL documents easier, and recall that such products were never
necessary in the IDL-era of distributed systems programming.

It is worth noting that REST systems \cite{fielding:rest} tend not to
use WSDL, even though it is theoretically possible. Instead they resort
to their XML type language of choice and quality human-readable
documentation. This would appear to be sub-optimal, yet REST is growing
in popularity, despite -or perhaps because of- the lack of WSDL
integration. 

Perhaps WSDL is not the appropriate language for
describing SOAP services; we are certainly not enthused about it. Yet
the sole solution being advocated is not a major undertaking to fix
WSDL's core flaws; it is to continue to encourage developers to hand
over to their SOAP stacks the challenge of deriving a stable and portable 
service interface from the inherently unstable and unportable sevice
implementation.   







