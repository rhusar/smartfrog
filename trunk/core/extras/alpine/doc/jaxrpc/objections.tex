\section{The Fundamental Flaws of JAX-RPC}
\label{objections:top}

\subsection{The O/X Mapping Problem}
\label{objections:o-x}

O/X mapping\footnote{Object-XML mapping} attempts to turn an XML
document into Java classes derived from the type information attached
to nodes. This is distinct from the type of mapping performed by
DOM\footnote{Document Object Model} in that the classes are
``serialised'' from the XML tree, not merely created to represent it. 
O/X mapping is an essential part of JAX-RPC, allowing method calls to
be translated into SOAP requests, and responses translated back into
Java objects.

We chose the term \emph{O/X mapping problem} to emphasise the
similarities it has with the heavily studied O/R mapping
problem\footnote{ \emph{"Object-relational mapping is the Vietnam of
Computer Science"} - Ted Neward. He means that it's a problem you can
never declare victory over, merely invest more and more effort in to get
deeper into the quagmire.
%http://www.neward.net/ted/weblog/index.jsp?date=20041003#1096871640048
}.  There is less experience in mapping between XML and objects, and
rather than drawing on the experiences of the many failed attempts at
O/R mapping problem, O/X mapping technologies appear destined to share
a similar evolution.

At first glance, O/X mapping appears simple: create a Java object
for each XML element, building a DAG\footnote{directed, acyclic graph}
when serializing to rpc/encoded SOAP or a tree with doc/lit
messages. Read or write between attributes and class fields, bind to
children and the conversion is complete. If only it were so
straightforward. Here are just some of the problems:

\todo{Ideally, each of these is discussed in a separate subsection
the list should be adapted to be exhaustive. We can omit detail
discussion of any that are especially obvious.}

\begin{itemize}

\item deciding which classes to use for an endpoint, and
deciding which XML elements to use for a Java class. 

\item deciding which types to use for the XML encoding. In rpc/enc messages,
this type must be explicitly declared from the small set of possible types. In
doc/lit messages, the type is omitted in messages. 

\item generating the WSDL description of service types through introspection.

\item map from XML names to valid Java names.

\end{itemize}

We discuss some of these points in the following sections.

\subsubsection{XML Elements and Java Classes}
\label{objections:o-x:xml-classes}

One particular problem is that the language of XML Schema is much
richer than the object model of Java. In Java, inheritance can extend
a type, and change some existing semantics, but derivation by
restriction is not explictly supported. Java, in common with many
object oriented programming environments, allows derived types to
expand upon the capabilities of their parents, but not to decrease
those capabilities.

In the XML schema language, it is natural to declare that a dog is a
mammal with four legs, and that a three legged dog is a restriction
where the number of legs must be three. In Java this type of
relationship cannot be explicitly modelled. \tocheck{As a more concrete
example, consider declaring that a {\tt Shape} has a {\tt getArea()}
method, and that a {\tt Point} is a {\tt Shape} without such a method}.

\todo{This section would benefit from a discussion of why the
fundamental differences in the type systems of the two representations
mean that mappings, if achieved, will always be unstable. Some of the
enumeration discussion below could migrate in here, but the
enumeration discussion doesn't perhaps go far enough.}

\subsubsection{Mapping XML QNames to Java Identifiers}
\label{objections:o-x:names}

\todo{Need to discuss the particular difficulties involved in mapping
XML QNames to Java identifiers, this will set the scene for the
following discussion of enumerations}.

One specific example that deserves special mention is how
\verb$xsd:enumeration$ declarations are mapped to Java. Prior to
Java1.5, there was no explict \verb$enum$ clause in the language, so
workarounds were developed. The JAX-RPC solution is that of a common
pattern: to declare a class with a public static instance representing
a valid enumeration value. The WSDL to Java code generates such a
class, with the name of each static class taken from the name of each
value in the enumeration.

This is a simple example of how O/X mapping should work. Except, what if the
value of the one of the enumeration types is a reserved word, such as in the
lifecycle state machine we wrote:

\begin{verbatim}
<xsd:simpleType name="lifecycleStateEnum">
  <xsd:restriction base="xsd:string"> 
    <xsd:enumeration value="initialized"/> 
    <xsd:enumeration value="running"/> 
    <xsd:enumeration value="failed"/> 
    <xsd:enumeration value="terminated"/> 
    <xsd:enumeration value="null"/> 
  </xsd:restriction>
</xsd:simpleType>
\end{verbatim}

One element in the enumeration is reserved, \verb|null|, so the
enumeration must now enumerate all states as \verb|value1|,
\verb|value2|, and so on, for the entire list. This means the
enumeration names in the Java source no longer contains any value at
all, other than the position number in the set. This is inordinately
brittle, as any change to the enumeration could reorder the values,
without the code detecting a change.

%%%%%%%%%%%%%%%%%%%%%%%%%%%%%%%%%
%pull this; no longer valid
% 
% \subsection{the mystery of unsigned long}
% 
% There is an \verb|unsignedLong| datatype in XML Schema. The JAX-RPC1.0 documents
% blissfully ignored this fact. Why so? Presumably because there is no way to
% represent an unsigned long in a simple java datatype. 
% 
% The implication of this absence is that there was no guarantee that a Java
% application can make SOAP requests of a service which uses unsigned long
% numbers. This is unfortunate, as there is nothing to stop a C++ or .NET SOAP
% endpoint from using such numbers. This was a foundational barrier to
% cross-platform interoperability. 
% 
% It has appeared in the JAX-RPC1.1 specification, for which we must be grateful,
% although we are still unable to use it our service for fear of a version1.1
% client attempting to talk to us

%%%%%%%%%%%%%%%%%%%%%%%%%%%%%%%%%

\subsection{WSDL: an extra complication}
\label{objections:wsdl-gen}

The role of an interface definition language (IDL) has always been
twofold:

\begin{enumerate}
\item Firstly, an IDL allows the creation of a definition of \emph{the
interface} of the remote system, independent of any particular
implementation, programming language or environment. This is
``interface'' in the sense of \emph{the implementation independent
signature of the service}, and does not imply that an implementation
language needs an explicit notion of interfaces. The interface is
inherently implementation independent, and can be frozen or carefully
managed with respect to versioning.

\item Secondly, the act of writing an IDL description forces the author to
define the system in terms of the portable datatypes and operations available in
the restricted language of the IDL. This can effectively guarantee portability,
and is a significant improvement over interfaces defined in the implementation
languages themselves, which invariably contain constructs which are not
portable.

\end{enumerate}

IDLs have many advantages for creating interoperable systems, yet the
generally accepted practice for working with JAX-RPC discards all
these notions. Instead of generating implementation classes from WSDL,
the WSDL description is usually generated from the implementation
classes using tools leveraging Java's Reflection API. We shall term
this process \emph{contract-last development}.

This has the following consequences.

\begin{itemize}

\item
    
There is no way to ensure that the published contract of a service
remains constant over time. Every redeployment of the service, every
upgrade of the SOAP stack or even the underlying Java runtime may
change the classes, and hence the contract.

\item

Some aspects of the service are not extracted from the raw signatures
of the classes and methods. For example, if a method chooses to
extract attachments from a message, that information can be hidden in
the contents of the message, instead of in the signature of the
call. The generated WSDL will hence omit any information about the
attachment needs of the service.

\item

There is no warning of portability issues before integration time.
When defining a service using an IDL, the author typically knows when
there are problems as the IDL will not compile. Yet with contract-last
development, everything may well seem to work until the service goes
live and a customer using a different language attempts to import the
WSDL and invoke the service.
    
\end{itemize}

The alternative to contract-last development is clearly \emph{contract-first
development}. Although this is the better approach from the perspective
of portability and interface stability, web service developers are not
pushed in this direction.

One of the underlying causes of this has to be the sheer complexity of
XML Schema and WSDL. The XSD type system bears minimal resemblance to
that of current object oriented languages, and WSDL itself is
over-verbose and under-readable. As evidence of this, consider the
broad variety of products that aim to make authoring XSD and WSDL
documents easier, and recall that such products were never necessary
in the IDL-era of distributed systems programming.

We note in passing that that REST systems \cite{fielding:rest} tend not to
make use of WSDL, even though it is theoretically possible. Instead
they resort to their XML type language of choice and quality
human-readable documentation. This would appear to be sub-optimal, yet
REST is growing in popularity, despite (or perhaps because of) the
lack of WSDL integration.

Returning to the desiderata for SOAP, following a contract-last process
sacrifices interoperability for ease of service development. Perhaps
WSDL is not the appropriate language for describing SOAP services (we
are certainly not enthused about it), yet the sole solution being
advocated is not a major undertaking to fix WSDL's core flaws, it is
to continue to encourage developers to hand over to their SOAP stacks
the challenge of deriving a stable and portable service interface from
the inherently unstable and unportable service implementation.

We are not proposing any changes to WSDL, merely mourning the fact
that its over-complexity discourages contract-driven
development more aggressively than any previous IDL ever did. We do
observe that once the type declarations of a service have been moved
into their own document, WSDL becomes much more manageable and this is
a pattern of service definition which we strongly encourage.


\subsection{SOAP is not RMI}
\label{soap-not-rmi}

JAX-RPC suffers from a greater flaw than those classically associated
with RPC invocation: it tries to make the communications look like
Java RMI. Java RMI is a simple and effective mechanism for connecting
Java classes running on different machines. It is an IDL-free
communication mechanism, which relies on introspection to create proxy
classes and to marshall classes. It works because the systems at both
ends are Java, usually different pieces of a single larger
application.

With synchronized versions of common code, objects can be trivially
serialized and sent over a network connection. Exceptions are just
another type of object, and so too can be sent over the wire. There is
no need for an IDL, as Java interface declarations can perform much of
the same role. And as the recipient is a remote object, state is
automatic.

JAX-RPC tries to reuse many of the programming patterns of RMI. For
example, the runtime will attempt to serialize classes marked as
\verb|serializable|, ignoring those fields marked as
\verb|transient|. It will even serialize complex compound objects where
possible. The user appears to have a reference to something like an
object, though one that represents the current conversation with an
endpoint, not a direct endpoint proxy.

When a message gets delivered to its endpoint, the normal handler for
dispatching the message creates an instance of the relevant Java class
and then dispatches the request to it. The message is turned into an
invocation of an instance of a remote object, which certainly looks
like RMI, even if the lifecycle of the class is radically different.

%% I want to find a new home for these paragraphs but I can't decide
%% where it is yet
%%
% We can see the consequence of this metaphor on the mailing list of
% users of Apache Axis. All too often, someone is confused by the errors
% displayed when they try and do something that they expect to work,
% such as send an low level class such as \verb|java.io.File| instance,
% or a \verb|javax.swing.JPanel|. The latter is serializable in RMI, yet
% clearly does not make sense for an interoperable web service.
%
% While it would be easy to laugh at their naiive and unrealistic
% expectations, we have to recognise that these users have been led to
% such expectations through a combination of marketing "Web Services!
% They're easy!" and the fact that JAX-RPC creates the illusion of being
% just like RMI. It is often even relatively easy to write the code that
% looks like a SOAP service, and then try and deploy it, only to find at
% runtime that the introspection system rejects Java classes as
% unacceptable, or the serialization logic raises deserialization
% exceptions when inbound messages are processed. A quick perusal of the
% Apache Axis user
% list\footnote{http://mail-archives.apache.org/eyebrowse/SummarizeList?listId=209}
% will demonstrate that these (along with deployment failures) are the
% most common problems

\subsubsection{Blocking RPC is often inappropriate}

On a fast network, RPC invocation is often a good choice as a
communication paradigm. Other models of communication are harder to
code, and the benefits are not apparent.

It is only when working over long-haul connections, or with large
content (eg fifteen megabyte attachments) that the limitations
of RPC patterns become clear. The only option we currently have is to
split network communication into a separate thread from any
interactive GUI thread, keeping the front end usable. While this keeps
the front end responsive, JAX-RPC provides no way to give the user
effective feedback or control over the communications. Specifically,
there is no way to receive progress notifications or cancel an active
call, even though the underlying transport code invariably permits
such features.

\subsubsection{Fault processing}

JAX-RPC tries to marshall Java faults over the network, in such a way
that they can be reconstituted at the far end into the same
fault\footnote{Before we criticise faults, we must note that one of
the authors has his name in the source of the {\tt AxisFault} fault
class, so may not be without blame himself.}. To manage this is a
somewhat complex process, as the classname of the fault must be
exchanged as the fault code.

We believe that attempting to seamlessly marshall faults is a mistaken
approach.  It propagates the still controversial "declare all possible
faults" rule of Java into remote interfaces, and so exposes platform
implementation details into the service interface. If a service could
only raise a normal {\tt SOAPFault}, or developers were forced to
declare and then implement custom WSDL fault elements, then a service
definition would be platform-neutral.

As well as the marshalled faults, the specification includes one
standard fault, {\tt SoapFaultException}.

\subsection{Implications}

What are the implications of these problems?

Let's put those people writing web services into two categories: those
who know and want to work with XML, and those who don't. 

JAX-RPC only benefits those who do not want to work with XML, by hiding
all the details, giving developers a model of remote method calls of
serialized java graphs. We have argued that this is a mistake, 
as it leads to O/X mapping problems, and retains an invocation model
that is inappropriate to long-distance networks or slow communications. 

JAX-RPC may hide the details of the XML, but instead present the users
with serialization errors, interoperability problems, and the challenge
of trying to maintain version consistency when the interface is derived
solely from introspection of the implementation classes.


For users who do want to work with XML, JAX-RPC is inappropriate because
it hides everything. Trying to integrate custom XML documents with
JAX-RPC serializations is possible, but very hard work and somewhat
inefficient, as DOM trees get recreated when assigning or extracting DOM
trees from SoapMessageElement implementations -at least with Apache
Axis.

There is one further problem with the JAX-RPC design and open source
projects. By isolating all the XML in the implementation, it creates a
big gulf between the exceedingly complex implementation code -low level
SAX and DOM processing, code generation from WSDL metadata, WSDL
generation from introspection, fault processing- and the end user code.
This makes it hard for end users to become developers, which is exactly
what open source platforms want. Open source projects succeed from the
contributions of the users, so depend upon users who are able and
willing to understand the implementation. JAX-RPC, by its very nature,
places too hard a boundary between end users and developers.

For the developers of proprietary implementations, this is not an issue,
and the inherent complexity of the implentation results in a product the
vendors can charge for. It is hard to get serialization,
interoperability, WSDL processing and introspection right, and then
there is the performance issue. Perhaps this is why both webMethods GLUE
and the BEA JAX-RPC implementations have a reputation for being faster
and more reliable than the open source version. 

It can only get worse. JAX-RPC has moved to become the centerpiece of
enterprise Java, in which RMI- and CORBA- based distributed
communications are considered legacy protocols. That is not by itself a
bad thing, but we believe that migration to Web Service should be done
from a position of understanding the XML and the message. Instead, more
and more developers will feel obliged to write Web Service accessors to
their code, and in doing so, fall into the traps that JAX-RPC creates
for them.

\subsubsection{The Future: JAX-RPC 2.0}

The forthcoming 2.0 release of the JAX-RPC specification promises to
correct some of these flaws, but does it correct enough? An alternate
O/X mapping is used: JAX-B\footnote{Java Architecture for XML
Binding}, which is independent of the SOAP Stack, but still represents
a compile-time declaration of what XML is to be expected. The 2.0
release also retains the core metaphor of service invocation as method
invocation, with the payload of most invocations being Java objects
that are somehow mapped to XML content. The automated generation of
WSDL from Java source is retained, despite this problem having been
shown to be hard.

We understand the rationale for much of this. Working with raw XML is
hard.  Writing good XML Schema documents is hard. WSDL is exceedingly
painful to work with\footnote{We'd be tempted to say it's like an
interface definition language, only more complex and
verbose!}. However, we believe that if service developers do not go to
the effort of writing their XSD and WSDL definitions of messages, then
they will never have control of the format of the messages that get
sent over the wire. Without that control, interoperability and
XML-based processing will be hard. Instead the developers will be left
at the mercy of JAX-RPC and JAX-B.

% \subsection{XSD validation}
%
% An XML schema document, as used to describe a doc/lit payload, can
% contain many declarations about what constitutes a valid XML
% document. \emph{Restrictions} can restrict the type of a value, such
% as the range of an integer from zero to ten. The \verb|minOccurs| and
% \verb|maxOccurs| declarations are used to declare what allowed number
% of occurences of an element are. None of these values are
% used. Restricted integer types are just mapped to normal integers,
% while element occurence declarations are ignored.

% This means that the set of XML documents which an endpoint can
% receive, is significantly larger than the set of XML documents which
% its XML schema will declare that it can support. With no built-in
% validation, developers must either write validation logic themselves
% -and corresponding tests, or ignore the problem. Given that there is
% no warning that the problem occurs, we suspect that ignorance is the
% common process. Not writing the logic to recognise and reject illegal
% documents is not likely to show up in development -especially if a
% test-centric process is not followed- but is inevitable once a service
% goes live, and callers using other languages invoke the service. Such
% insidious defects, defects that only show up in production, are never
% welcome.

% We note that there is a trivial solution to this problem, one that is
% common to other XML stacks. It is: validate incoming messages against
% the XML schema of the service. This is not done for performance, yet
% the effect is to hide defects.
