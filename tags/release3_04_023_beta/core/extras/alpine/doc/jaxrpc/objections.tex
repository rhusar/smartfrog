\section{The Fundamental Flaws of JAX-RPC}
\label{objections}

\subsection{The Object/XML Impedance Mismatch}
\label{objections:o-x}

JAX-RPC attempts to turn an XML document into Java classes, using
service specific mapping information. This is distinct from the kind
of mapping performed by DOM implementations, in that the classes are
``serialised'' from the XML tree, not merely created to represent it
(it is a semantic rather than syntactic mapping). This
serialisation/deserialisation is an essential part of JAX-RPC,
allowing method calls to be translated into SOAP requests, and
responses translated back into Java objects.

We believe that the term \emph{serialisation} downplays the nature of
the problem, likening it to the more tractable problem of creating a
non-portable persistence format for a class. Instead, we prefer to use
the term \emph{O/X mapping} to emphasise the similarities it has with
the heavily studied \emph{O/R mapping problem}\footnote{
\emph{``Object-relational mapping is the Vietnam of Computer Science''}
- Ted Neward, 2004.
%{\small \tt http://www.neward.net/ted/weblog/index.jsp?date=20041003#1096871640048} 
}.  Over a decade has been spent trying create robust mappings between
records in relational databases and language-level objects, and there
is still no sign of an ideal solution. There is significantly less
experience in mapping between XML and objects, and rather than drawing
on the experiences of the many failed attempts at O/R mapping, O/X
mapping technologies appear destined to share a similar evolution.

At first glance, the O/X mapping problem facing JAX-RPC appears
simple: create a Java object for each XML element, building a
directed, acyclic graph when serialising to RPC/encoded SOAP or a tree
when using document/literal messages. Read or write between attributes
and class fields, bind to children and the conversion is
complete... If only it were so straightforward. Undermining it all is
a fundamental difference between the type systems of XML (especially
that of XML Schema) and that of Java, making any mapping both complex
and brittle.

\subsubsection{Binding XML Elements to Java Classes}
\label{objections:o-x:xml-classes}

The language of XML Schema is much richer than the object model of
Java. In Java, inheritance can extend a type, and change some existing
semantics, but derivation by restriction is not explicitly
supported. Java, in common with many object oriented programming
environments, allows derived types to expand upon the capabilities of
their parents. XML schema allows one to extend a type by restricting
it, constraining attribute and element values. Java has no intrinsic
model for this type of constraint.

This is a fundamental difference which means that one cannot
accurately model an XSD type hierarchy in a Java class hierarchy. All
one can do is inaccurately model it. Here, for example, a postcode is
modelled by restricting a string:

%\todo{more entertaining example. something like `us-beer' is a 
%restriction of type `beer' where alcohol is a removed attribute}

\begin{verbatim}
<simpleType name="UKPostcode">
  <restriction base="xsd:string">
    <pattern value="[A-Z]{2}\d \d[A-Z]{2}"/>
  </restriction>
</simpleType>
\end{verbatim}

The actual result is going to be a simple class of type {\tt String}:
all restriction information will be lost in the transformation from
XML Schema to Java. This is a fundamental difference, and one which
would appear to remain intractable except in special cases.

Note that XML Schema offers other type extension mechanisms, such as
substitution and derivation. These mechanism have similar issues with mapping
to the inheritance and type model of Java. 

\subsubsection{Mapping XML Names to Java Identifiers}
\label{objections:o-x:names}

Not all XML names can be turned into Java identifiers.  XML names may
begin with a letter in one of many Unicode languages, an ideograph or
an underscore (``\_'') . They can be followed by any of the same
characters, and also a hyphen ``-'' or a full stop ``.''. Some
examples are: {\tt schr\"odinger}, {\tt \_unknown.type-set}, and {\tt
String}.

Java identifiers almost comprise a proper subset of XML
names\footnote{XML names beginning in ``xml'' (any case) are reserved, whereas in
Java they are valid identifier names.}.
Because of the much greater range of allowable XML identifiers, the
system will often need to perform a non-trivial mapping from the XML
names to valid class and package names. Package names are typically
derived from namespace URLs if not overridden, as discussed in section
\ref{objections:o-x:namespaces}.

The translation is inordinately brittle: whenever a new
version of Java is released, the logic must be updated to avoid new
reserved words (like {\tt assert} and {\tt enum}), or the generated
code will no longer compile in the enhanced language. Needless to say,
such an upgrade will break any existing code that linked to old
classes which made use of these names.

\subsubsection{Enumerations}
\label{objections:o-x:enum}

One specific example that deserves special mention is how {\tt
xsd:enumeration} declarations are mapped to Java. Before Java 1.5,
there was no explicit {\tt enum} clause in the language, and the
JAX-RPC approach to SOAP enumerations is based on a
workaround. However, the problem of mapping enumerations from XML to
Java is unchanged, regardless of the language version used: generate a
set of identifiers, one for each value in the enumeration.

This appears a straightforward example of how O/X mapping should
work. But what if the value of the one of the enumeration types is a
reserved word?  Our API (as described in section
\ref{intro:experience} contains a lifecycle state machine like this:

\begin{verbatim}
<xsd:simpleType name="lifecycleStateEnum">
  <xsd:restriction base="xsd:string"> 
    <xsd:enumeration value="initialized"/> 
    <xsd:enumeration value="running"/> 
    <xsd:enumeration value="failed"/> 
    <xsd:enumeration value="terminated"/> 
    <xsd:enumeration value="null"/> 
  </xsd:restriction>
</xsd:simpleType>
\end{verbatim}

One element in this enumeration is reserved: {\tt null}. However, the
JAX-RPC specification states that an implementation must now enumerate
all states as {\tt value1}, {\tt value2}, and so on, for the entire
list.  The enumeration names in the Java source no longer contain any
informative value at all, other than a position number in the set. Any
change to the enumeration could reorder the values, without this
change being detected by code that used the enumeration. The defect
would only show up in interoperability testing.

\subsubsection{Unportable types}
\label{objections:o-x:types}

Some Java types are by nature explicitly unportable. One would not
expect to be able to have a SOAP runtime serialise a database
connection instance and have it reconstituted in working order at the
far end, for example. One might hope that a {\tt java.util.Hashtable}
could be translated into some XML structure that could be turned into
a platform-specific equivalent at the far end. But surely a {\tt
java.util.Calendar} object could be sent over the wire, with its
obvious relationship to the {\tt xsd:dateTime} type in XML Schema?

We can certainly attempt to send such times. They are readable on the
wire, and are mapped into whatever the remote endpoint uses to
represent time. Unfortunately, in this case, differences in
expectations between Java and .NET date/time classes prevent the same
time being received at the far end. If both client and server are in
the UTC time zone all works well, but if either of them are in a
different location, hours appear to get added or removed. Clearly a
different expectation regarding time processing is at work.

This is an insidious defect as it is not apparent on any
testing which takes place in the same time zone, or between Java
implementations. It is only apparent when remote callers, using
different platforms, attempt to use the service.


\subsubsection{Serialising a Graph of Objects}
\label{objections:o-x:graphs}

XML is a hierarchical data structure, and can only describe trees or
lists of data. Java classes almost invariably refer to other objects,
often creating cyclic graphs of references. If such a cyclic graph is
to be mapped into XML, the mapping infrastructure must recognise the
cycle (a naive implementation would enter a non-terminating
loop). Once the cycle is recognised, it must be addressed. The options
appear to be:
\begin{enumerate}
\item Signal an error.
\item Insert cross references into the XML message, for processing by
the mapper at the destination. 
\item Break the graph by duplicating content in the XML.
%% This doesnt differ from 2 in any obvious way?
%\item Insert explicit relationship information into the message.
\end{enumerate}

The only one of these solutions which seamlessly marshals cyclic
graphs of objects is the second: inserting cross references into the
XML. The SOAP solution for this is described in section 5 of the
specification \cite{spec:SOAP1.1}. This linking mechanism is only
supported in RPC/encoded messages; the document/literal message format
does not allow it.

JAX-RPC was originally based on RPC/encoded messages, but the
alternate representation, document/literal, is now broadly agreed to
be more flexible and generally superior. There is no way to marshal a
cyclic graph into a document/literal message without custom
code\footnote{This problem is covered in detail in ``Effective
Enterprise Java'' \cite{neward:EEJ}, where it is termed
\emph{the object-hierarchical impedance mismatch}.}. Any technology
that attempts to map XML to a cyclic object graph will suffer from the
same problem.

\subsubsection{XML Metadata and Namespaces}
\label{objections:o-x:namespaces}

As discussed in the previous sections, XML Schema provides a type
system that is much richer than that of Java. One aspect not mentioned
so far is the relationship between XML metadata, notably namespaces,
and Java classes.

The problem is essentially as follows: each node in an XML message can
have attached to it a namespace. There is no related construct in Java
which can model this accurately. The choice that is normally made is
to model it inaccurately by package names (mapping namespaces to Java
packages provides many of the problems discussed in section
\ref{objections:o-x:names}, since these are more examples of identifiers). 

The problems that typically arise are of two kinds:

\begin{enumerate}

\item Mapping an incoming message to a web service object requires
that either the namespace of either the operation itself or its
parameters be guessed. This guessing can be wildly inaccurate when the
web service's Java interface was generated from WSDL using package
renaming.

\item When dynamic invocation is desired (service invocation without
the use of pre-built stub classes) it can be very difficult to
determine the correct namespaces for service invocations (the WSDL
typically leaves this unspecified, meaning that for JAX-RPC services
the WSDL is not a complete description of the service interface).

\end{enumerate}

If more metadata were recorded with generated types, this problem would not
arise. We therefore believe that future versions of JAX-RPC will address this 
problem by way of the annotation facility recently added to the Java
language. 

\subsubsection{Message validation}
\label{objections:o-x:validation}

When a message is received, the serialised form is generated and
passed to the handlers for processing. In typical SOAP stacks,
no validation of the incoming XML against the message schema is
performed, and in particular any restrictions on the number of times
an item is required are not checked. This forces the implementation
code to follow one of two paths:
\begin{enumerate}

\item It could ignore the problem. If the client code and functional
tests do not generate invalid messages (as is likely if they are also
all written in JAX-RPC) then the problem will not be noticed, only
only surfacing when a third party attempts to use the service.

\item The developers could write procedural logic to verify that the
Java classes representing a deserialised message have a structure that
matches their expectations given the schema. This requires an
understanding of the schema, knowledge of the serialisation mapping
and its potential trouble spots, and the willingness to write the tests to
validate this extra logic.

\end{enumerate}

We suspect that most services err on the side of ignorance, and do not
validate their incoming messages adequately. This brings their ability
to operate in a heterogeneous environment into serious question.

\subsubsection{Inadequate Mixing of XML and Serialised Data}
\label{objections:o-x:mixing}

JAX-RPC and JAXM are two different views of the world. When JAX-RPC
encounters a piece of XML which cannot be deserialised within a
message, it creates a SAAJ {\tt Node} to describe that part of the
document tree.  From that point on, the tree below the node is
permanently isolated from JAX-RPC processing (in some sense the
developer has sailed off the edge of the JAX-RPC world, and fallen
into the universe of XML.) Any O/X mappings which may exist for data
within this piece of the message are now inaccessible: all that is
left is the low-level JAXM API.

This behaviour implies that incorporating arbitrary XML within a SOAP
message is not an approved action, yet the ability to easily
incorporate such XML is a key aspect of SOAP's flexibility and a major
part of enabling it to be more extensible and less brittle than its
predecessors.

\subsubsection{Fault processing}
\label{objections:soap-not-rmi:faults}

JAX-RPC's fault handling encompasses two distinct and opposing
viewpoints. On one hand, the {\tt SOAPFaultException} type provides
direct access to the XML elements of a generic SOAP fault. Users of
this mechanism can construct SOAP faults with arbitrary
contents\footnote{In Apache Axis, this technique is used to add
non-standard diagnostic elements such as hostname, HTTP error codes
and stack trace to messages.} and process faults from any remote
endpoint, regardless of its implementation type or version.

In contrast, the other fault processing mechanism offered by JAX-RPC
aims to provide for loss-less marshalling of Java exceptions across
the SOAP infrastructure. The implementation and platform dependence
inherent in this represents the complete antithesis of the generic
SOAP fault mechanism.

While this may seem a simpler approach for the application developer,
mapping faults into Java exceptions with a specific static structure
means that extra data included in the original SOAP fault is likely to
be lost. For example, should a remote runtime include a hostname or
stack trace, this information will be stripped out on the receiving
side before the client can be made aware of it. The process is also
inordinately brittle: if, for example, an updated endpoint were to
throw a new exception, any callers built against earlier WSDL would
have to revert to handling a generic {\tt SOAPFaultException}.

While the goal appears to have been to provide near-seamless
marshalling of Java exceptions whilst permitting the exceptions to be
immutable, the technical details of the marshalling undermine the
success of the approach even without the interoperability problems.
For example, there is a requirement that special constructors exist to
configure the created exception to preserve its immutability, and WSDL
generation is unreliable unless sufficient debug information is
included in the bytecode for the names of all parameters to the
constructor to be visible.

We believe that the attempt JAX-RPC makes to seamlessly marshal faults
is consistent with its general approach, but like so much of that
approach, it undermines the interoperability and robustness of
services created with it. By propagating Java's requirement that all
posssible exceptions be declared into remote interfaces, it exposes
platform implementation details. Interoperability is a major strength
of SOAP, yet here JAX-RPC tries too hard to mimic Java RMI, and
abandons many of the gains SOAP would otherwise have over RMI. In our
opinion, the only sensible approach for working with faults in JAX-RPC
is to have services throw pure SOAP faults and leave it to the
recipient to process them.

\subsection{SOAP is not just RPC}
\label{objections:soap-not-just-rmi}

SOAP's parentage includes XML-RPC \cite{winer:xmlrpc} and (indirectly) COM/DCOM
\cite{dbox:com} and CORBA \cite{vinoski:CORBA}. It was clearly designed at its
outset to be a form of remote procedure call in XML, over HTTP. Over time, the
world-view that lead to that choice has changed. Though it is often presented as
a form of RPC, we would argue that it is coming to be seen as more powerful when
viewed as a system where arbitrary XML documents are exchanged between parties,
perhaps asynchronously, and potentially via intermediaries.

In this world, the programming paradigms that seemed appropriate for
an RPC infrastructure look out of place. On a fast network, RPC
invocation is often a good choice, as other models of communication
are harder to code, and their benefits are not readily apparent. A complex
communication can be modelled in a few lines of code, rather than a state
machine, and the synchronous nature of the communication makes it easier to
to build a model of the state of the remote system.  

When we begin to work over long-haul connections, however, or with
large content (e.g.\ several megabyte attachments), the limitations of
RPC become clear. The greatest of these is that RPC is
synchronous. Although asynchronous behaviours can, with some
difficulty, be introduced, this is not the natural way for RPC to
behave. As content becomes larger and the network latency increases,
the problems posed by synchronous calls become more and more acute.

Currently, our only option is to split network communication into a
separate thread from the rest of the program. While this works, it
provides the programmer no way to give the user effective feedback or
control over the communications. There is no way to receive progress
notifications or cancel an active call, despite the face that the
underlying transport code invariably permits such features. This can
cause problems when working with file transfers, foe example: one of
the authors wrote a GUI front end to a service that could accept
15-30MB CAD files, and whilst multithreading could keep the UI
responsive, there was no way to present an upload progress indicator
or offer a cancel button. These are both features one expects in an
application of this kind.

Again, following our principle that SOAP technologies should uphold
the same desiderata as SOAP itself, we note that one reason SOAP was
adopted was to simplify the task of working over long haul
connections. By making it both difficult and complicated to work over
a long connection, JAX-RPC fails to meet this criterion for a SOAP
technology.

\subsection{SOAP is not RMI}
\label{objections:soap-not-rmi}

JAX-RPC suffers from a greater flaw than those classically associated
with RPC invocation: it tries to make the communications look like
Java RMI. Java's RMI system is a simple and effective mechanism for
connecting Java classes running on different machines. It is an
IDL-free system that relies on introspection to
create proxy classes and to marshal data. It works because the
systems at both ends are running on the Java platform, typically
different pieces of a single larger application. Even then, it's
implementation-first design means that it is brittle to change, and
most effective when both ends are using the same versions of all
classes.

In a system with shared code at both ends, objects can be trivially serialised
and transmitted across a network connection. Exceptions are just another type of
object, and so too can be sent over the wire. There is less need for an IDL, as
Java interface declarations can perform much of the same role. And as the
recipient is a remote object, state is automatic. One can even keep code
synchronised by using a special class loader, one that fetches code from jointly
accessible URLs.

JAX-RPC tries to reuse many of the programming patterns of RMI. For
example, the runtime will attempt to serialise classes marked as {\tt
Serializable}, ignoring those fields marked as {\tt transient}. It
will even serialise complex compound objects where possible. The user
appears to have a reference to something like an object, though one
that represents the current conversation with an endpoint, not a
direct endpoint proxy.

Recall that one of our key hopes in adopting SOAP (section \ref{introduction})
was to enable loose coupling between the components of a distributed system.
SOAP strove to overcome many of the failings of precursor technologies like
CORBA and DCOM. These technologies worked over local area networks, and
enabled rich bidirectional communications, but were not cross
platform\footnote{Admittedly for arguably political rather than technical
reasons}, and ended up being used to produce distributed object systems that
were too tightly coupled.

While Java RMI provides convenience, the one thing it does not provide
in any way is loose coupling. Interacting systems typically run from
the same codebase, and each element of the distributed system contains
many implicit assumptions about the rest of the system. By trying to
turn SOAP into RMI, we imitate this architecture, and risk losing the
very things we turned to SOAP for in the first place.

% Tight coupling has proven to be a bad thing in a distributed
% system. If components are too tightly coupled, it implies that the
% separate components have merged to become one large system, one which
% cannot be treated except as a whole. And in a sufficiently large
% system, it is impossible to co-ordinate the whole \cite{deutsch,
% JINI}. The move towards SOAP and a service oriented architecture was
% advocated as a means of dealing with loose organisational coupling, by
% forcing a loose data coupling between applications.

\subsection{WSDL: an extra complication}
\label{objections:wsdl-gen}

The role of an interface definition language (IDL) has always been
twofold:

\begin{enumerate}
\item Firstly, an IDL allows the creation of a definition of \emph{the
interface} of the remote system, independent of any particular
implementation, programming language or environment. This is
``interface'' in the sense of \emph{the implementation independent
signature of the service}, and does not imply that an implementation
language needs an explicit notion of interfaces. The interface is
inherently implementation independent, and can be frozen or carefully
managed with respect to versioning.

\item Secondly, the act of writing an IDL description forces the author to
define the system in terms of the portable datatypes and operations available in
the restricted language of the IDL. This can effectively guarantee portability,
and is a significant improvement over interfaces defined in the implementation
languages themselves, which invariably contain constructs which are not
portable.

\end{enumerate}

IDLs have many advantages for creating interoperable systems, yet the
generally accepted practice for working with JAX-RPC discards all
these notions. Instead of generating implementation classes from WSDL,
the WSDL description is usually generated from the implementation
classes using tools leveraging Java's Reflection API. We shall term
this process \emph{contract-last development}.

This has the following consequences.

\begin{itemize}

\item
    
There is no way to ensure that the published contract of a service
remains constant over time. Every redeployment of the service, every
upgrade of the SOAP stack or even the underlying Java runtime may
change the classes, and hence the contract.

\item

Some aspects of the service are not extracted from the raw signatures
of the classes and methods. For example, if a method chooses to
extract attachments from a message, that information can be hidden in
the contents of the message, instead of in the signature of the
call. The generated WSDL will hence omit any information about the
attachment needs of the service.

\item

There is no warning of portability issues before integration time.
When defining a service using an IDL, the author typically knows when
there are problems as the IDL will not compile. Yet with contract-last
development, everything may well seem to work until the service goes
live and a customer using a different language attempts to import the
WSDL and invoke the service.
    
\end{itemize}

The alternative to contract-last development is clearly \emph{contract-first
development}. Although this is the better approach from the perspective
of portability and interface stability, web service developers are not
pushed in this direction.

One of the underlying causes of this has to be the sheer complexity of
XML Schema and WSDL. The XSD type system bears minimal resemblance to
that of current object oriented languages, and WSDL itself is
over-verbose and under-readable. As evidence of this, consider the
broad variety of products that aim to make authoring XSD and WSDL
documents easier, and recall that such products were never necessary
in the IDL-era of distributed systems programming.

We note in passing that that REST systems \cite{fielding:rest} tend not to
make use of WSDL, even though it is theoretically possible. Instead
they resort to their XML type language of choice and quality
human-readable documentation. This would appear to be sub-optimal, yet
REST is growing in popularity, despite (or perhaps because of) the
lack of WSDL integration.

Returning to the desiderata for SOAP, following a contract-last process
sacrifices interoperability for ease of service development. Perhaps
WSDL is not the appropriate language for describing SOAP services (we
are certainly not enthused about it), yet the sole solution being
advocated is not a major undertaking to fix WSDL's core flaws, it is
to continue to encourage developers to hand over to their SOAP stacks
the challenge of deriving a stable and portable service interface from
the inherently unstable and unportable service implementation.

We are not proposing any changes to WSDL, merely mourning the fact
that its over-complexity discourages contract-driven
development more aggressively than any previous IDL ever did. We do
observe that once the type declarations of a service have been moved
into their own document, WSDL becomes much more manageable and this is
a pattern of service definition which we strongly encourage.



