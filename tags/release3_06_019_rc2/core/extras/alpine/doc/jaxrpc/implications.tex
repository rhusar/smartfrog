\section{Implications}
\label{implications}

We believe that only two categories of web service developer exist:
those who are comfortable with XML and want to work with it, and those
who aren't but end up doing so anyway. JAX-RPC provides a sugar coated
wrapping that encourages developers who are relatively unfamiliar with
XML to bite. Yet, as anyone who has written a web service of any
complexity knows, the XML must be faced and understood eventually. In
practise, the task of creating a real web service is made more
difficult, not less, by the huge volume of code JAX-RPC introduces
into a project.

JAX-RPC only superficially benefits developers who do not want to work
with XML: by hiding all the details, and giving developers a model of
remote method calls via serialised Java graphs, JAX-RPC makes it
harder to write true, interoperable SOAP services. Not only that, but
it introduces the O/X mapping problem, while retaining an invocation
model that is inappropriate for long-distance networks and slow
communications.

We argue that JAX-RPC greatly complicates users' software by
introducing a complex and fickle serialisation system. The generation
of WSDL from Java code, which JAX-RPC encourages, makes it very
difficult to maintain version consistency of an interface, and
creates significant interoperability problems.

On top of all of this, for users who do want to work with the XML
(typically those whose first project did not!) JAX-RPC is
inappropriate because it hides everything. Trying to integrate custom
XML documents with JAX-RPC serialisations is possible, but very hard
work. In Apache Axis, DOM trees get recreated when assigning or
extracting them from {\tt SoapMessageElement} implementations.


\subsection{The Future}
\label{implications:future}

JAX-RPC has become a cornerstone of Enterprise Java
\cite{spec:J2EE-14}, alongside RMI and RMI-over-CORBA. That is not by
itself a bad thing, but we believe that it creates the misconception
that developers can trivially migrate from RMI to SOAP. If they
attempt to do, they will fall into the traps that JAX-RPC creates for
them.

The 2.0 revision of the JAX-RPC specification promises to correct some
of these flaws, but we believe many problems remain. It uses a new O/X
mapping, the Java Architecture for XML Binding 2.0 (JAXB
2.0)\cite{spec:JAX-B-20}. This is a sophisticated framework for
mapping XML into Java, having broad support for the generation of
classes from XML Schema, and of XML Schema from annotated classes. 
Despite many improvements over its predecessors, we believe JAXB is
simply a more comprehensive approach to a problem that is
fundamentally ill-posed.
  
JAX-RPC 2.0 continues the attempt to provide a metaphor of
method-invocation on object instances across SOAP transport. 
This fundamentally unmaintainable metaphor, coupled with the
automated generation of WSDL from Java source code, means that in our
opinion JAX-RPC has not outgrown its limitations. 

It is easy to understand the rationale behind the decisions made in
designs JAX-RPC and JAXB 2.0. Working with raw XML is hard. Writing
good XML Schema documents is hard. WSDL is exceedingly painful to work
with. Asynchronous messaging is more complex than blocking
RPC. Nevertheless, we believe that all these things will be part of
any solution enabling developers to realise the promised
interoperability and loose coupling of SOAP-based systems.

WSDL itself is also evolving, with the 2.0 revision of the
specification nearing completion \cite{spec:WSDL-20}. It is to be
hoped that a new WSDL syntax will make it easier to specify service
contracts, and some new features of the language (for example the {\tt
interface} construct) promise to increase the opportunities for reuse
and extension of service interfaces. Unfortunately, the sheer breadth
of the WSDL goal set makes it exceedingly complex, and this, coupled
with the needless verbosity that it enforces, is likely to make it a
difficult language to work with for some time\footnote{As any SOAP
stack developer will attest, bug reports involving handwritten WSDL
are the ones they fear.}.


