\section{Introduction}
\label{introduction}

%\PARstart{I}{n}
In
beginning any discussion of SOAP-based technologies, it is 
valuable to review the core features which made adopting SOAP
attractive initially. We briefly characterise these as follows:

\begin{enumerate}
\item Simplicity: It is intended to be easy to work with.

\item Interoperability: It is more interoperable than binary
predecessors.

\item Extensibility: The envelope/header/body structure
allows extra data to be attached to a request, potentially without
breaking existing systems.

\item Self-describing: Messages can contain type definitions
alongside data, and provide human readable names.

\item Flexibility: Participants can handle variable amounts
of incoming data.

\item Long-haul: It is designed to work through firewalls,
building upon HTTP and other established protocols.

\item Loosely-coupled: Participants are not expected to share
implementation code.

\item XML-centric: Built on XML and intended to integrate with
XML-based technologies.

\end{enumerate}

We will refer to these criteria throughout our discussion, as the
desiderata against which any SOAP technology should be judged.

\subsection{SOAP in Java}
\label{intro:java}

Communication with SOAP can be viewed either as XML-based remote
procedure calls, or as a way of submitting XML documents to remote
endpoints (optionally eliciting responses in the form of XML
documents.) These two different perspectives represent the RPC-centric
and message-centric viewpoints. In Java, the RPC-centric model has
become the primary model for SOAP APIs.

The Java APIs representing the two different underlying perspectives
are JAXM\footnote{Java API for XML Messaging}
\cite{spec:JAX-M-11} and JAX-RPC\footnote{Java API for XML-based RPC}
\cite{spec:JAX-RPC-11}. We review each of these in turn.

\subsubsection{JAXM}
\label{intro:jaxm}

JAXM was written to support both basic SOAP, and more complex
scenarios like asynchronous ebXML message exchange over SOAP. This
flexibility introduces significant extra complexity into the design.
Over time, the ebXML focus of JAXM has declined, while the API itself
has been renamed SAAJ\footnote{SOAP with Attachments API for Java}
\cite{spec:SAAJ-12}.

In JAXM/SAAJ, the developer works with the SOAP message through Java
interfaces derived from DOM\footnote{Document Object Model}
\cite{spec:DOM}. These are bound to a class that represents the body
of the message, which provides various operations to manipulate the
pieces. These include accessors and manipulators for the envelope,
headers, body and any binary attachments.

JAXM does not provide significant transport support: the primary
method of receiving JAXM messages is to implement and deploy a HTTP
servlet.  The sole method of sending a message is to ask a
{\tt SOAPConnectionFactory} for a {\tt SOAPConnection} instance, and
then make a blocking call of the endpoint.

JAXM has become an orphan specification. Had ebXML been more
successful, it is conceivable that JAXM might have proven more
popular, and made message-centric SOAP development in Java
commonplace. As it is, JAX-RPC is touted as the recommended way to
work with SOAP in Java.

\subsubsection{JAX-RPC}
\label{intro:jax-rpc}

In JAX-RPC\footnote{The current edition of the JAX-RPC specification
is version 1.1, and this is the version we discuss here. JAX-RPC 2.0
is discussed in section \ref{implications:future}.} the XML
representation of a message's encoding is hidden and developers work
with Java objects created automatically from the XML data using a
semi-standardised mapping process. Java classes can be automatically
turned into SOAP endpoints, with each public method in the class
exported as an operation with both a request message and a response
message. The message structure is described in a WSDL descriptor and
an associated XML Schema, both of which can be hand-written, or
automatically extracted from the Java classes through introspection.

Client side proxy classes can be generated from the WSDL files, proxy
classes which again provide a method for every operation the service
supports. In communications between systems running on the Java
platform, the result is that methods called on the proxy class result
in the passing of the method parameters to remote methods on an
instance of the implementation class, a behaviour that superficially
resembles Java RMI \cite{paper:RMI}. We will return to this in section
\ref{objections:soap-not-rmi}.

One effective architectural feature of JAX-RPC is the \emph{handler chain},
which consists of an ordered sequence of classes which are configured
to manage requests and responses. Using the handler chain, it is
possible to add support for new SOAP headers to existing services, or
to apply extra diagnostics, in a relatively transparent fashion. The
dispatch of operations to Java methods, Enterprise Java Bean methods
or other destinations is generally implemented by specific handlers,
making the handler chain the foundation for the rest of the system.

JAX-RPC is widely implemented, both by open source projects (for
example Apache Axis \cite{apache:axis}), and by commercial vendors
like Sun, IBM and BEA. Although these SOAP toolkits do all implement
the appropriate version of JAXM/SAAJ to complement the RPC model, this
feature is neither broadly promoted nor used. All the
\emph{evangelisation} of SOAP focuses on JAX-RPC, as do most of the
examples in the vendors' documentation.

The bias is such that, for Java development, it is widely seen that
JAX-RPC \emph{is} SOAP.

\subsection{The Hard Lessons of Service Implementation}
\label{intro:experience}

The authors have recently been involved in developing independent
implementations of a SOAP API for deployment \cite{draft:CDDLM}. This
API, specified in a set of XML Schema (XSD) \cite{spec:XSD} and Web
Services Description Language (WSDL) files \cite{spec:WSDL-11} defines
a service endpoint providing seven operations, which permit suitably
authenticated callers to deploy distributed applications onto a grid
fabric.

The development of this service was performed in a ``pure way'', by
creating the XSD and WSDL files first. This approach is believed to
aid in creating a platform-independent system, and represents current
best practise. However, the XSD file for the service messages is
approximately 2000 lines, including all the comments and annotations
needed to make it comprehensible. That it takes so many lines to
describe a relatively simple service is clearly one reason why this
approach is so unpopular.

Many problems were encountered turning this service specification into
functional clients and servers, problems that we attribute largely to
JAX-RPC. In section \ref{objections} we discuss a number of the
problems we believe this work highlighted. Section
\ref{objections:wsdl-gen} outlines the particular problems we believe
typical JAX-RPC oriented approaches to WSDL generation create.





