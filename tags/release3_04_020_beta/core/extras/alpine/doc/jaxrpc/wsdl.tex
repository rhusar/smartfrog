\subsection{WSDL: an extra complication}
\label{objections:wsdl-gen}

The role of an interface definition language (IDL) has always been
twofold:

\begin{enumerate}
\item Firstly, an IDL allows the creation of a definition of \emph{the
interface} of the remote system, independent of any particular
implementation, programming language or environment. This is
``interface'' in the sense of \emph{the implementation independent
signature of the service}, and does not imply that an implementation
language needs an explicit notion of interfaces. The interface is
inherently implementation independent, and can be frozen or carefully
managed with respect to versioning.

\item Secondly, the act of writing an IDL description inherently
forces the author to define the system in terms of the portable
datatypes and operations available in the restricted language of the
IDL.  This can effectively guarantee portability, and is a significant
improvement over similar definitions in implementation languages,
which invariably contain constructs which are not portable. As such
constructs are excluded from the interface language, a portability
issue is the exception, rather than being commonplace.
\end{enumerate}

IDLs have many advantages for creating interoperable systems, yet the
generally accepted practise for working with JAX-RPC discards all
these notions. Instead of generating implementation classes from WSDL,
the WSDL description is usually generated from the implementation
classes using tools leveraging Java's Reflection API. We shall term
this process \emph{contract-last development}.

This has the following consequences.

\begin{itemize}

\item
    
There is no way to ensure that the published contract of a service
remains constant over time. Every redeployment of the service, every
upgrade of the SOAP stack or even the underlying Java runtime may
change the WSDL, and hence the interface.

\item

Some aspects of the service are not extracted from the raw signatures
of the classes and methods. For example, if a method chooses to
extract attachments from a message, that information can be hidden in
the contents of the message, instead of in the signature of the
call. The generated WSDL will hence omit any information about the
attachment needs of the service.

\item

There is no warning of portability issues before integration time.
When defining a service using an IDL, the author typically knows when
there are problems as the IDL will not compile. Yet with contract-last
development, everything may well seem to work until the service goes
live and a customer using a different language attempts to import the
WSDL and invoke the service.
    
\end{itemize}

The alternative to contract-last development is clearly \emph{contract-first
development}. Although this is the better approach from the perspective
of portability and interface stability, web service developers are not
pushed in this direction.

One of the underlying causes of this has to be the sheer complexity of
XML Schema and WSDL. The XSD type system bears minimal resemblance to
that of current object oriented languages, and WSDL itself is
over-verbose and under-readable. As evidence of this, consider the
broad variety of products that aim to make authoring XSD and WSDL
documents easier, and recall that such products were never necessary
in the IDL-era of distributed systems programming.

We note in passing that that REST systems \cite{fielding:rest} tend not to
make use of WSDL, even though it is theoretically possible. Instead
they resort to their XML type language of choice and quality
human-readable documentation. This would appear to be sub-optimal, yet
REST is growing in popularity, despite (or perhaps because of) the
lack of WSDL integration.

Returning to the desiderata for SOAP, following a contract-last process
sacrifices interoperability for ease of service development. Perhaps
WSDL is not the appropriate language for describing SOAP services (we
are certainly not enthused about it), yet the sole solution being
advocated is not a major undertaking to fix WSDL's core flaws, it is
to continue to encourage developers to hand over to their SOAP stacks
the challenge of deriving a stable and portable service interface from
the inherently unstable and unportable service implementation.

We are not proposing any changes to WSDL, merely mourning the fact
that its over-complexity discourages contract-first, contract-driven
development more aggressively than any previous IDL ever did. We do
observe that once the type declarations of a service have been moved
into their own document, WSDL becomes much more manageable and this is
a pattern of service definition which we strongly encourage.
