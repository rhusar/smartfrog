\section{Alpine: a proposed alternative}
\label{alpine}

We are in the preliminary stages of designing an alternative SOAP
stack for Java, which we are calling \emph{Alpine}\footnote{Its inspiration
being derived in part from nimble, lightweight Alpine-style mountaineering
approaches.}. 

\subsection{Manifesto}
\label{alpine:manifesto}

Our goal is to create a SOAP stack that is easy to use, robust, and
maintainable. In order to do this, we are adopting an XML centric
approach. Alpine will make no attempt to map between XML and custom
Java classes, instead providing access to the SOAP messages using
modern XML support libraries, which make it easy to navigate an XML
document. By avoiding O/X mapping we greatly decrease the volume and
complexity of our code. Some may argue this will make Alpine more
difficult to use, but experience shows us that simpler systems are
typically more straightforward to work with, as they react in more
predictable ways.

If a WSDL/XSD description of an Alpine-hosted service is required, the
user will be required to write it: as we concluded in section
\ref{objections:wsdl-gen} generating these from Java introduces
unwanted implementation dependencies and hampers interoperability.

With so much stripped out, Alpine will be a SOAP stack reduced to its
essentials: a system for managing the flow of messages through a set
of handlers, and libraries to handle transport across supported the
protocols. Core compliance with the SOAP protocol will be provided,
namely envelope validation and \verb|mustUnderstand| processing of
headers. Developers will be expected to use XPath specifications to
work with contents of the message; we are considering basing our
design upon the ``XOM'' XML framework \cite{harold:xom}.

This will not be a SOAP stack that attempts to make SOAP look like
Java RMI, nor will it prevent developers from being aware of the
format of the messages sent over the wire. Instead, Alpine will just
provide the basic housekeeping and handler chain management to make
simplify web service development, leaving the interpretation and
mapping of the XML messages to the applications themselves.

\subsection{Design Goals}
\label{alpine:design}

The full design goals are as follows:

\begin{enumerate}

\item Stay in the XML space as much as possible.
\item Take advantage of as much leading edge infrastructure as we can.
\item Adopt the the handler chain pattern of Axis/JAX-RPC.
\item Target SOAP 1.2 (POST) only, WS-I Basic Profile 1.1 \cite{spec:WSI-11}.
\item Document/literal mssages only, not RPC/encoded.
\item Support XSD and Relax NG schemas.
\item Run server-side, client-side, and as an intermediary. 
\item No support for JAX-RPC or JAX-M/SAAJ APIs.
\item Configurable procedurally, through the Java Management API (JMX).
\item Permit dynamic handler chain configuration during message processing.
\item One supported parser. %: Xerces
\item Run on Java 1.5 and later.

\item No provision of side features such as a built in HTTP server, or
a declarative configuration mechanism. These are delegated to other products.
\end{enumerate}

We believe the core of this design is likely to resemble JAX-M/SAAJ in
in terms of classes, integrated with a handler chain based on the
JAX-RPC/Axis model.

\subsection{XSD validation}
\label{alpine:validation}

Although we are still unsure as to how complete our WSDL support will be, we
note that document/literal SOAP messages can be validated simply by comparing
the incoming messages to the XML Schema that describes them. Mainstream SOAP
stacks do not do this at present, usually for performance reasons, but this 
means that the set of XML documents an endpoint might receive is significantly 
larger than the set of those considered valid by its XML schema. With no automatic
validation, developers must either write both validation logic and corresponding
tests themselves, or choose to ignore the problem. Given that there is no warning 
of potential problems before deployment, we suspect that many developers remain 
unaware of the problems they face.

Errors caused by the absence of logic to detect and reject illegal
documents are unlikely to show up in development, especially if a
test-centric process is not followed, but become inevitable once a
service goes live, and callers using other languages invoke the
service. Such insidious defects, defects that only show up in
production, are always unwelcome.

There is a trivial solution to this problem, one that is common to
other XML stacks. It is: validate incoming messages against the XML
schema of the service.  We aim to implement a handler which will do
this, which, if included on a handler chain, will reject invalid
messages. It will also be able to validate outbound messages, which
should be useful during development. 

\subsection{A Community SOAP Stack}
\label{alpine:community}


An open source project, by its very nature, is written by its users, and so it 
is critical to such a project's success that as many users as possible are
able to contribute to its implementation. The split between
XML in JAX-RPC's internal API, and mapped object instances in the external API,
creates a gulf between the implementation of the stack, and that of end user
applications. This creates a significant barrier between external and internal 
developers, making it much more difficult for users to contribute to the system's
development. With Alpine, we hope to avoid such a split, because the XML emphasis, 
and hence the terms of reference, are consistent between user applications and
the SOAP stack itself.


\subsection{The Implications of Alpine}
\label{alpine:implications}

If Alpine succeeds, it will be a SOAP stack that requires an
understanding of XML before it can be used. This might appear to be a
barrier to the widespread adoption of the tool, and perhaps it will
prove so. Unlike commercial SOAP vendors, we have no financial incentive
to make our product broadly usable. We will, however, have a SOAP
implementation which all its users should be able to understand and
maintain.  Furthermore, we believe that a good understanding of XML is
needed to develop any robust web service, and that by forcing developers to
acquire that skill early on in the development process, we help them avoid
being forced to learn it just before their shipping deadlines are missed.

This may seem somewhat ruthless: to deny the right to write web
services to developers who are and wish to remain ignorant of XML.
However, we have to ask: \emph{if they do not want to know XML, why are
they writing web services?} If the developers want to use a less portable,
more brittle, remote method invocation system, they would be better off
using a stable technology such as Java RMI or CORBA.

If Alpine is not adopted, then either the design was flawed,
or it did not appeal to a sufficiently large developer community to survive. At
present it seems the design problem is the most pressing one. We have argued that JAX-RPC is the
wrong API for working with SOAP in Java. If an XML-centric design were to prove
equally unworkable, this might well mean that the promised flexibility of
XML messaging infrastructures is not easily accessible to languages of the ``Java 
generation'' (in which we include C\# and VB.NET), all of which share a similar
static type system and object model. Should these languages not prove flexible enough
to exploit the full potential of XML, then it may be that the promise of XML messaging 
systems, both REST and SOAP, will only be realised by the next generation of platforms, be they
extensions of existing languages, such as C$\omega$, or XML runtimes such as
Apache Cocoon and NetKernel by 1060 Research
\cite{MSFT:TransitionsInProgrammingModels,pjr:NKonXml}.
