\section{Implications}
\label{implications}

What are the implications of these problems?

We believe that there are only two categories of people writing web
services: those who know XML and want to work with it, and those who
don't but end up doing so anyway. JAX-RPC provides a sugar coated
wrapping that encourages developers who are relatively unfamiliar with
XML to bite. Yet, as anyone who has written a web service of any
complexity knows, the XML must be faced and understood eventually. In
practice, the task of creating a real web service is made more
difficult, not less, by the huge volume of code JAX-RPC introduces.

JAX-RPC only superficially benefits developers who do not want to work
with XML: by hiding all the details, and giving developers a model of
remote method calls via serialised Java graphs, JAX-RPC makes it harder
to write true, interoperable SOAP services. Not only that, but it
introduces the O/X mapping problem, while retaining an invocation model
that is inappropriate for long-distance networks and slow
communications.

We argue that JAX-RPC greatly complicates users' software
by introducing a complex serialisation system (which, at least in
Apache Axis, is notoriously fickle.) The generation of WSDL from Java
code, which JAX-RPC encourages, by makes it very difficult to
maintain version consistency of an interfaces, and creates significant
interoperability problems. 

On top of all of this, for users who do want to work with the XML
(typically those whose first project did not!) JAX-RPC is
inappropriate because it hides everything. Trying to integrate custom
XML documents with JAX-RPC serialisations is possible, but very hard
work. In Apache Axis, DOM trees get recreated when assigning or
extracting them from {\tt SoapMessageElement} implementations.

Anonther consequence of the way the API gives the illusion of remote method
invocation, could arguably be the appearance of specifications that present a
``resource'' view of the remote system, where a remote resource consists of 
attributes and operations: an object, by any other name. 
A programming paradigm that presents remote services as local objects, may have
created pressure to change the remote service model to match that local
representation. The result is specifications such as WS-ResourceFramework
\cite{spec:WSRF}.
 

% For the developers of proprietary implementations, this is not an
% issue, and the inherent complexity of the implentation results in a
% product the vendors can charge for. It is hard to get serialisation,
% interoperability, WSDL processing and introspection right, and then
% there is the performance issue. Perhaps this is why both webMethods
% GLUE and the BEA JAX-RPC implementations have a reputation for being
% faster and more reliable than the open source version.

\subsection{The Future}
\label{objections:implications:future}

JAX-RPC has become a cornerstone of Enterprise Java \cite{spec:J2EE-14}, in
which distributed communications based on RMI and CORBA are considered
legacy protocols. That is not by itself a bad thing, but we believe
that migration to web services should be done from a position of
understanding the XML and the message. Instead, more and more
developers will feel obliged to write Web Service accessors to their
code, and in doing so, fall into the traps that JAX-RPC creates for
them.

The forthcoming 2.0 release of the JAX-RPC specification promises to
correct some of these flaws, but does it correct enough? An alternate
O/X mapping is used: JAX-B\footnote{Java Architecture for XML
Binding}, which is independent of the SOAP Stack, but still represents
a compile-time declaration of what XML is to be expected. The 2.0
release also retains the core metaphor of service invocation as method
invocation, with the payload of most invocations being Java objects
that are somehow mapped to XML content. The automated generation of
WSDL from Java source is retained, despite this problem having been
shown to be fundamentally flawed.

We understand the rationale for much of this. Working with raw XML is
hard.  Writing good XML Schema documents is hard. WSDL is exceedingly
painful to work with. However, we believe that if service developers
do not write their XSD and WSDL definitions of messages, they will
never have control of the messages that get sent over the wire, and
without that control, interoperability and loose coupling will remain
out of reach.  

%Instead the developers will be left at the mercy of JAX-RPC and JAX-B.
