\begin{nusmvCommand} {show\_traces} {Shows the traces generated in a \nusmvhead session}

\cmdLine{show\_traces [-h] [-v] [-t] [-m | -o output-file] [-p plugin-no]
    \linebreak {[-a | trace\_number]}}

Shows the traces currently stored in system memory, if any. By default
it shows the last generated trace, if any.

\begin{cmdOpt}
\opt{-v} { Verbosely prints traces content (all state variables,
otherwise it prints out only those variables that have changed their
value from previous state). This option only applies when the Basic
Trace Explainer plugin is used to display the trace.}

\opt{-t}{ Prints only the total number of currently stored traces.}

\opt{-a}{ Prints all the currently stored traces.}

\opt{-m}{ Pipes the output through the program specified by the
\shellvar{PAGER} shell variable if defined, else through the \unix
command \shellcommand{more}.}

\opt{-o \parameter{\filename{output-file}}}{Writes the output generated by the command to
\filename{output-file}.}

\opt{-p \parameter{\natnum{plugin-no}}}{Uses the specified trace
  plugin to display the trace.}

\opt{\natnum{trace\_number}}{ The (ordinal) identifier number of the trace to
 be printed. This must be the last argument of the command. Omitting
 the trace number causes the most recently generated trace to be printed.}
\end{cmdOpt}

If the XML Format Output plugin is being used to save generated traces
to a file with the intent of reading them back in again at a later
date, then only one trace should be saved per file. This is because
the trace reader does not currently support multiple traces in one
file.

\end{nusmvCommand}
